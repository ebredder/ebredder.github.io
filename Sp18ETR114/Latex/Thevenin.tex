\documentclass[tikz]{article}
\usepackage[siunitx]{circuitikz}
\usepackage{siunitx}
\usepackage{graphicx}
\usepackage{subcaption}
\usepackage[margin=.5in]{geometry}
\usepackage{mathtools}

\begin{document}

  \section{Thevenin's Theorem}

Thevenin's Theorem: Any resistive network or circuit can be represented as a voltage source in series with a source resistance. This helps predict how the circuit will respond to a change in load.

\subsection{Thevenin Voltage}
The Thevenin Voltage $(V_{TH})$ of a circuit is the voltage present at the output terminal when the load is removed.
\subsubsection{Example}

  \begin{circuitikz} \draw
  (0,0) node[ground]{}
  (0,0) to [battery,l=12<\volt>] (0,8)
  (0,8) to [resistor, l=20 <\ohm>] (4,8)
  to [resistor, l=100 <\ohm>] (4,4)
  to [resistor, l=360 <\ohm>] (4,0)
  (4,4) node[circ]{}
  (4,4) -- (8,4)
  (8,4) node[ocirc]{}
  to [american potentiometer, l=$R_L$] (8,0)
  (8,0) node[circ]{}
  (8,0) node[ground]{}
  (4,0) node[ground]{}
;
  \end{circuitikz}

  \begin{circuitikz} \draw
  (0,0) node[ground]{}
  (0,0) to [battery,l=12<\volt>] (0,8)
  (0,8) to [resistor, l=20 <\ohm>] (4,8)
  to [resistor, l=100 <\ohm>] (4,4)
  to [resistor, l=360 <\ohm>] (4,0)
  (4,4) node[circ]{}
  (4,4) -- (8,4)
  (8,4) node[ocirc]{}
  (8,0) node[ocirc]{}
  (8,0) node[ground]{}
  (4,0) node[ground]{}
;
  \end{circuitikz}

$$V_{TH}=V_S\frac{R_3}{R_T}=(12V)\frac{360\Omega}{480\Omega}=9V$$

\subsubsection{Problem}
Find the Thevenin Voltage $(V_{TH})$ of the circuit below:

\begin{circuitikz} \draw
(0,0) node[ground]{}
(0,0) to [battery,l=3.3<\volt>] (0,8)
(0,8) to [resistor, l=6 <\ohm>] (4,8)
to [resistor, l=33 <\ohm>] (4,4)
to [resistor, l=27 <\ohm>] (4,0)
(4,4) node[circ]{}
(4,4) -- (8,4)
(8,4) node[ocirc]{}
to [american potentiometer, l=$R_L$] (8,0)
(8,0) node[ground]{}
(4,0) node[ground]{}
;

\end{circuitikz}

\subsection{Thevenin Resistance}
The Thevenin Resistance $(R_{TH})$ is the resistance measured across the output terminals when the load is removed. To determine this circuit we will also note the voltage source will be replaced by a wire.
\subsubsection{Example}

\begin{circuitikz} \draw

  (0,0) to [battery,l=12<\volt>] (0,4)
  to [resistor, l=30 <\ohm>] (4,4)
  to [resistor, l=10 <\ohm>] (4,0)
  (4,4) to [resistor, l=15 <\ohm>] (8,4)
  (8,0) to [resistor, l=$R_L$] (8,4)
  (8,0) -- (0,0)
  (4,0) node[ground]{}
  ;
\end{circuitikz}

To measure the Thevenin Resistance we need to remove the power supply and replace it with a wire. The load is produced by measuring the resistance with your multimeter.

\begin{circuitikz} \draw

  (0,0) -- (0,4)
  to [resistor, l=30 <\ohm>] (4,4)
  to [resistor, l=10 <\ohm>] (4,0)
  (4,4) to [resistor, l=15 <\ohm>] (8,4)
  (8,0) node[ocirc]{}
  (8,4) node[ocirc]{}
  (8,0) -- (0,0)
  (4,0) node[ground]{}
  ;
\end{circuitikz}

$$R_{TH} = (R_1 || R_2) + R_3 = 30 \Omega || 10 \Omega + 15 \Omega = 7.5\Omega + 15 \Omega = 22.5 \Omega$$

\section{Norton's Theorem}

Thevenin's Theorem: Any resistive network or circuit can be represented as a current source in parallel with a source resistance. This helps predict how the circuit will respond to a change in load.

\subsection{Norton Voltage}
The Thevenin Voltage $(V_N)$ of a circuit is the current present at the output terminal when the load is removed and shorted.
\subsubsection{Example}

\begin{circuitikz} \draw
(0,0) node[ground]{}
(0,0) to [battery,l=3.3<\volt>] (0,8)
(0,8) to [resistor, l=6 <\ohm>] (4,8)
to [resistor, l=33 <\ohm>] (4,4)
to [resistor, l=27 <\ohm>] (4,0)
(4,4) node[circ]{}
(4,4) -- (8,4)
(8,4) node[ocirc]{}
to [american potentiometer, l=$R_L$] (8,0)
(8,0) node[ground]{}
(4,0) node[ground]{}
;
\end{circuitikz}

\begin{circuitikz} \draw
(0,0) node[ground]{}
(0,0) to [battery,l=3.3<\volt>] (0,8)
(0,8) to [resistor, l=6 <\ohm>] (4,8)
to [resistor, l=33 <\ohm>] (4,4)
to [resistor, l=27 <\ohm>] (4,0)
(4,4) node[circ]{}
(4,4) -- (8,4)
(8,4) node[ocirc]{}
to [ammeter] (8,0)
(8,0) node[ground]{}
(4,0) node[ground]{}
;
\end{circuitikz}

\subsection{Norton Resistance}
The Thevenin Resistance $(R_N)$ is the resistance measured across the output terminals when the load is removed. To determine this circuit we will also note the voltage source will be replaced by a wire.

\end{document}
